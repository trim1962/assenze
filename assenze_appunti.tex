% !TXS template
% !TeX TS-program = lualatex
\documentclass{article}
% \usepackage{fontspec}
\usepackage{amsmath}
\usepackage[a4paper]{geometry}
\usepackage{babel}
\begin{document}
\section*{Assenze}
La normativa prevede che: 	DPR n. 122 del 22 giugno 2009 art. 14 c. 7 che recita

\begin{quotation}

 “[…]ai fini della validità dell’anno scolastico
[…] per poter procedere alla valutazione finale di ciascun studente, 
è richiesta la frequenza di almeno tre
quarti dell’orario annuale personalizzato”.
\end{quotation}

la Circolare n.20 Roma, 4 marzo 2011 spiega cosa è l'orario annuale personalizzato

	\begin{quotation}
		[..]
	Personalizzazione del monte ore annuo
	L’art. 11 del decreto legislativo n. 59/2004 e i richiamati articoli 
	2 e 14 del Regolamento
	parlano espressamente di “orario annuale personalizzato”.
	A riguardo è opportuno precisare che tali riferimenti devono essere interpretati per la scuola secondaria di primo grado 
	alla luce del nuovo assetto ordinamentale 
	definito dal d.P.R. 20 marzo
	2009 n. 89 (in particolare dall'art. 5) e, per la scuola 
	secondaria 
	di secondo grado, in relazione alla
	specificità dei piani di studio propri di ciascuno dei percorsi 
	del nuovo o vecchio ordinamento
	presenti presso le istituzioni scolastiche.
	L’intera questione della personalizzazione va, comunque, inquadrata
	 per tutta la scuola secondaria nella cornice normativa del d.P.R. 275/99 e,
	 in particolare, degli artt. 8 e 9 del predetto regolamento.
	Pertanto devono essere considerate,a tutti gli effetti, 
	come rientranti nel monte ore annuale del curricolo di ciascun 
	allievo tutte le attività 
	oggetto di formale valutazione intermedia e finale da
	parte del consiglio di classe 
	[..]
	\end{quotation}
Qui abbiamo il primo problema se un alunno non si avvale della religione cattolica 
bisogna togliere le 33 ore annuali di religione.

Il monte ore annuale è quindi se poniamo che vi siano 33 settimane di 32 ore

\[monte\; ore=33\times 32=1056\]
mentre il monte ore per chi non avvale è ponendo che vi siano 33 ore di religione

\[monte\; ore=33\times 32-33=1023\]

Le conseguenze di questo è che l'alunno che non si avvale ha una percentuale di assenze, a parità di numero, maggiore rispetto a quello che si avvale. Esempio Pierino e Carlo hanno 200 ore di assenza, Pierino si avvale Carlo no quindi
\[Assenze\; Pierino=\dfrac{200}{1023}\times 100=20\%\]
\[Assenze\; Carlo=\dfrac{200}{1056}\times 100=19\%\]

\section*{Assenze Spaggiari}
Il registro elettronico Spaggiari  ha tre modi diversi per calcolare le assenze per gli alunni. Il coordinatore ha accesso a questi tramite il percorso

Coordinatore $-->$ Stampe Registro $-->$ Assenze 
\begin{description}
	\item[Su assenze e presenze] Vengono rilevate le ore di presenza, le ore di assenza e la percentuale di assenza è \[PA=\dfrac{Assenze}{Assenze+Presenze}\times 100\]
	\item[Su monte ore annuali] Le ore di assenza vengono divide per il monte ore annuali \[PA=\dfrac{Assenze}{1056}\times 100\]
	\item[Su monte ore attuale] Le ore di assenza vengono divide per il monte ore attuale cioè per il totale delle ore che la classe ha attualmente svolto  \[PA=\dfrac{Assenze}{Monte\; ore\; attule}\times 100\]
\end{description}
Facciamo alcuni esempi pratici

A oggi, 16 maggio abbiamo in una classe, questa situazione:

\begin{center}
	\begin{tabular}{|l|c|c|c|c|c|c|c|}
	\hline
Alunno	& P & A &  P+A &Moa & \%PeA & \%MO & \%MoA   \\
	\hline
1	& 506 & 130& 636 & 951 & 20 & 12 & 14   \\
	\hline
2	&763  &64&827  & 951 &7  & 6 &  7  \\
	\hline
3	&627  &200&827  & 951 &24  & 19 &  21  \\
	\hline
4	&680  &145&825  & 951 &17  & 14 &  15  \\
\hline
\end{tabular} 
\end{center}
Aggiungiamo che 1 è arrivato durante l'anno scolastico. La percentuale per quanto esprime la normativa è MO perché riferita al mote ore annuale.
\end{document}
