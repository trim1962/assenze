% !TXS template
% !TeX TS-program = lualatex
\documentclass{article}
% \usepackage{fontspec}
\usepackage{amsmath}
\usepackage[a4paper]{geometry}
\usepackage[italian]{babel}
\usepackage[style=italian]{csquotes}
\usepackage[style=philosophy-modern]{biblatex}
\addbibresource{assenze.bib}
\usepackage{hyperref}
\begin{document}
\section*{Assenze}
La normativa prevede che: DPR n. 122 del 22 giugno 2009 art. 14 c. 7 che recita

\begin{quotation}

 “[…]ai fini della validità dell'anno scolastico
[…] per poter procedere alla valutazione finale di ciascun studente, 
è richiesta la frequenza di almeno tre
quarti dell'orario annuale personalizzato”.
\end{quotation}
Quindi è errato e fuorviante parlare di giorni d'assenza.
La Circolare n.20, 4 marzo 2011 spiega cosa è l'orario annuale personalizzato:

	\begin{quotation}
		[..]
	Personalizzazione del monte ore annuo
	L’art. 11 del decreto legislativo n. 59/2004 e i richiamati articoli 
	2 e 14 del Regolamento
	parlano espressamente di “orario annuale personalizzato”.
[..], per la scuola 
	secondaria 
	di secondo grado, in relazione alla
	specificità dei piani di studio propri di ciascuno dei percorsi 
	del nuovo o vecchio ordinamento
	presenti presso le istituzioni scolastiche.
	L’intera questione della personalizzazione va, comunque, inquadrata
	 per tutta la scuola secondaria nella cornice normativa del d.P.R. 275/99 e,
	 in particolare, degli artt. 8 e 9 del predetto regolamento.
	Pertanto devono essere considerate,a tutti gli effetti, 
	come rientranti nel monte ore annuale del curricolo di ciascun 
	allievo tutte le attività 
	oggetto di formale valutazione intermedia e finale da
	parte del consiglio di classe 
	[..]
	\end{quotation}
Qui abbiamo il primo problema: se un alunno non si avvale della religione cattolica, 
dobbiamo togliere le 33 ore annuali della materia.

Il monte ore annuale è quindi, se poniamo che vi siano 33 settimane di 32 ore

\[monte\; ore=33\times 32=1056\]
mentre il monte ore per chi non avvale è, ponendo che vi siano 33 ore di religione,

\[monte\; ore=33\times 32-33=1023\]

Le conseguenze di questo è che l'alunno che non si avvale ha una percentuale di assenze, a parità di numero, maggiore rispetto a chi non segue questa materia.

 Esempio Pierino e Carlo hanno un totale 200 ore di assenza, Pierino si avvale mentre Carlo no quindi
\[Assenze\; Pierino=\dfrac{200}{1023}\times 100=20\%\]
\[Assenze\; Carlo=\dfrac{200}{1056}\times 100=19\%\]

\section*{Assenze Spaggiari}
Il registro elettronico Spaggiari  ha tre modalità diverse per calcolare le percentuali delle assenze degli alunni. Il coordinatore ha accesso a queste informazioni tramite il percorso:

Coordinatore $-->$ Stampe Registro $-->$ Assenze 

Le modalità sono:
\begin{description}
	\item[Su assenze e presenze] Il registro elettronico tiene conto delle ore di assenza e presenza di un alunno  in base a quello che inserisce l'insegnante durante la firma appello, quindi un errore, una mancata firma etc. sono causa di  calcoli non corretti \[PA=\dfrac{Assenze}{Assenze+Presenze}\times 100\]
	\item[Su monte ore] Le ore di assenza vengono divise per il monte ore annuale \[PA=\dfrac{Assenze}{1056}\times 100\]
	\item[Su monte ore oggi] Le ore di assenza vengono divise per il monte ore attuale cioè per il totale delle ore che la classe ha attualmente svolto. Queste ore sono calcolate dal registro utilizzando l'inserimento dell'orario settimanale, orario annuale ed eventuali chiusure per forza maggiore.   \[PA=\dfrac{Assenze}{Monte\; ore\; attule}\times 100\]
\end{description}
Facciamo alcuni esempi pratici

A oggi, 16 maggio abbiamo in una classe, questa situazione:

\begin{center}
	\begin{tabular}{|l|c|c|c|c|c|c|c|}
	\hline
Alunno	& P & A &  P+A &Moa & \%PeA & \%MO & \%MoA   \\
	\hline
Uno	& 506 & 130& 636 & 951 & 20 & 12 & 14   \\
	\hline
Due	&763  &64&827  & 951 &7  & 6 &  7  \\
	\hline
Tre	&627  &200&827  & 951 &24  & 19 &  21  \\
	\hline
Quattro	&680  &145&825  & 951 &17  & 14 &  15  \\
\hline
\end{tabular} 
\end{center}
Aggiungiamo che Uno è arrivato da un'altra scuola ad anno scolastico già iniziato. Purtroppo capita, che non sono note le assenze dell'altra scuola e questo falsa il conteggio finale.
\section*{La gestione dei certificati}
La normativa che regola le assenze prevede delle deroghe e degli sconti. Tali riduzioni influiscono sul numero delle assenze e sulla loro percentuale. 
Per esempio un alunno ha 420 ore di assenza con una percentuale pari a $40\%$ circa. L'alunno presenta una dichiarazione in cui afferma che è stato assente per un mese causa ricongiungimento familiare. Come vanno scontate le ore in maniera corretta? 
La normativa parla di orario quindi nel calcolo vanno tolti: i sabati, le domeniche ed eventuali altre festività o sospensioni dato che in questi giorni non vi è stata scuola. Se il mese in questione ha quattro settimane e inoltre la scuola è stata chiusa un mercoledì per la festività del santo patrono abbiamo che le ore complessive di scuola sono \[4\times 32=128\]
A questo conteggio va tolto il mercoledì di chiusura. Se per il giorno era un rientro si toglieranno otto ore, altrimenti sei, quindi 
\[128-8=120\]
\[128-6=122\]
Le sue ore di assenza dopo il mese in famiglia saranno
\[420-120=300\]
altrimenti
\[420-122=298\]
le sue percentuali
sono \[\dfrac{300}{1056}\times 100=28\%\]  oppure \[\dfrac{298}{1056}\times 100=28\%\] In entrambi casi l'alunno è bocciato.

Altro esempio un alunno è assente da mercoledì prima di Pasqua e torna quello successivo. L'alunno presenta un certificato di sette giorni per malattia ma potrà scontarne solo uno quindi avrà sei ore in meno.
\section*{Sospensioni}
Le sospensioni  vanno conteggiate escludendo nel recupero i giorni di chiusura della scuola. la nota 3602 del 31 luglio 2008 - Modifiche apportate allo Statuto delle studentesse e degli studenti recita:
\begin{quotation}
	[..]
	C) Sanzioni che comportano l’allontanamento temporaneo dello studente dalla comunità scolastica per un periodo  superiore a 15 giorni (Art. 4 – Comma 9).
	[..]
	[..]
	D) Sanzioni che comportano l’allontanamento  dello studente dalla comunità scolastica fino al termine dell'anno scolastico ( Art. 4 - comma 9bis):
	[..]
	[..]
	Con riferimento alle sanzioni di cui ai punti C e D, occorrerà evitare che l’applicazione di tali sanzioni determini, quale effetto implicito, il superamento dell'orario minimo di frequenza richiesto per la validità dell'anno scolastico. Per questa ragione dovrà essere prestata una specifica e preventiva attenzione allo scopo di verificare che il periodo di giorni per i quali si vuole disporre l’allontanamento dello studente non comporti automaticamente, per gli effetti delle norme di carattere generale, il raggiungimento di un numero di assenze tale da compromettere comunque la possibilità per lo studente di essere valutato in sede di scrutinio.
	[..]
\end{quotation}
\nocite{*}
\printbibliography
\end{document}
